\documentclass[12pt]{article}
\usepackage{amsmath}
\usepackage{graphicx}
\usepackage[utf8]{inputenc}
\usepackage[T1]{fontenc}
\usepackage[french]{babel}
\usepackage{amsfonts,amsmath,amssymb}
\usepackage{float}
\usepackage{graphicx}
\usepackage{amsmath}
\usepackage{newunicodechar}
\usepackage[export]{adjustbox}
\usepackage[a4paper]{geometry}
\usepackage{fancyhdr}
\pagestyle{fancy}
\renewcommand\headrulewidth{1pt}
\fancyhead[L]{Documentation interactive dans les applis du rectorat}
\fancyhead[R]{Académie de Rennes}
\renewcommand\footrulewidth{1pt}
\fancyfoot[L]{Mouhyi MOUTARAJJI}
\begin{document}
\begin{titlepage}
	
    \vspace*{0.0 cm}
    \begin{flushleft}
    
    \includegraphics  [scale = 0.6] {diagrammes/logo_Rectorat.jpg}\includegraphics  [scale = 0.1] {diagrammes/lo.png}
    \end{flushleft}
    \textsc{96 Rue d'Antrain}\hspace{100 pt}\textsc{ UFR Informatique - Électronique }\\
    \hspace{20 pt}\textsc{35700 RENNES }\hspace{200 pt}\textsc{Campus De Beaulieu}\\
   \text{.} \hspace{289 pt}\textsc{35042 Rennes Cedex}\\
   \\
   \\
   \\
   \\
 	\centering   % Course Code
	\textsc{\large Master 2 Informatique}\\
		\textsc{\large Ingénierie Logicielle en alternance}\\
		\textsc{\large 2018/2019}\\
	\rule{\linewidth}{0.7 mm} \\[0.2 cm]
	\textsc{\large Documentation interactive dans les applications du rectorat}\\
	\rule{\linewidth}{0.2 mm} \\[0.2 cm]

	\centering   % Course Code
	\textsc{ Mouhyi MOUTARAJJI}\\
		\begin{flushleft} \large
		    \emph{Maîtres d'apprentissage:} \\
		    Annabel BOURDÉ, Annabel.Bourde@ac-rennes.fr\\
			Gaël SALAÜN , Gael.Salaun@ac-rennes.fr\\
			
			
		\end{flushleft}
	
\end{titlepage}


\newpage

\tableofcontents
~
\newpage

\section{Introduction}

J'ai réalisé mon alternance de Master 2 Ingénierie logiciel au sein du rectorat de l'académie de rennes du 09 septembre 2018 au 30 août 2019. J'ai intégré la DSII (Direction des Systèmes d'Information et de l'Innovation) et plus précisément le pôle DINAMO (Développement et INtégration en environnement Académique ou Mutualisé) encadré par Annabel BOURDÉ et Gaël SALAÜN. Ce pôle s'occupe généralement de développer et maintenir des applications web national utilisées par les chefs des établissement, les élèves  ou leurs parents et aussi des applications métiers dédiés aux employés du rectorat.\newline


Durant mon alternance, j'ai eu pour principale mission d'intégrer un outil de documentation interactive dans les applications du rectorat. Le travail consistait à trouver une solution qui permet aux administrateurs des applications d'être autonomes et pouvoir rajouter la documentation sans utiliser du code informatique. Nous avons réaliser cette solution ainsi réaliser une interface graphique utilisée par ces administrateurs, j'ai donc eu la chance de pouvoir développer aussi bien sur le back-end que sur le front-end.\newline   

C'était une première expérience en tant qu'apprenti dans une entreprise, cela m'a permis d'acquérir des nouvelles méthodes de travail comme l'agilité et de plus j'ai pu travaillé sur des des concepts innovants ainsi qu'utiliser des technologies et outils actuels.\newline


Dans un premier temps, j'établirai une bref présentation du service informatique du rectorat en présentant les multiples pôles qui existent et son organisation. Puis je présenterai plus en détail les objectifs de mon alternance ainsi que les méthodologies de gestion de projet que j'ai suivies et les outils que j'ai utilisés. Enfin j'expliquerai les travaux réalisés dans ma mission et je conclurai avec un bilan de mon alternance.

\newpage


   

\section{Contexte général}
\subsection{Rectorat de l’académie de Rennes}

Le rectorat de l'académie de Rennes est une circonscription administrative propre à l’Éducation Nationale.\\
Elle regroupe 4 directions des services départementaux de l'Éducation Nationale : 
Côtes d'Armor, Finistère, Ille-et-Vilaine et Morbihan. Chaque direction des services départementaux de l'Éducation Nationale est placée sous la responsabilité d'un inspecteur d'académie, directeur académique des services de l'Éducation Nationale. Son rôle au sein de ces départements est de gérer les personnels, les élèves, les moyens financiers, les examens... de tous les établissements scolaires, depuis la maternelle jusqu'au lycée.


\subsection{Organisation}
(TODO)

\subsection{Pôle DINAMO}

Au sein du rectorat, nous trouvons le pôle DINAMO (Développement et INtégration en environnement Académique ou Mutualisé), dans lequel je travaille en tant qu'apprentis et dont l'activité principale est le développement et l'intégration académiques. 


\begin{enumerate}
\item \textbf{Missions :}\\

\begin{itemize}
\item Mettre en œuvre des services numériques académiques ou mutualisés.
\item Réalisation des applications web et fournir des applications fiables, sécurisées et adaptées aux usages des utilisateurs. 
\item Assurer la maintenance et l'évolution des projets.\\
\end{itemize}

\item \textbf{Effectifs :}\\
L’équipe de développement du pôle DINAMO est constituée d’Arnaud Rupin, responsable du pôle, ainsi que 6 développeurs informatiques (Albane GUIHOMAT, Annabel BOURDÉ, Gaël SALAÜN, Maïna DIRINGER, Marc BERHAUT et Vincent CARON). Voir figure 1\\

\end{enumerate}

\begin{figure}[H]
	\centering
 		\includegraphics[width=1\textwidth]{diagrammes/PoleDinamo.png}
  		\caption{l'équipe du pôle DINAMO}
	\end{figure}


\subsection{Méthodes de travail}
\subsubsection{Méthodologie agile}

les nouveaux projets de l'équipe DINAMO fonctionnent en méthodologie agile sous forme de sprints qui varie de 2 à 4 semaines. Chaque équipe effectue un Daily-meeting afin d'échanger sur l'évolution des tâches à faire.

Durant un sprint, l'équipe embarque des nouvelles \textbf{User Story} et réaliser des nouvelles fonctionnalités et aussi corriger les erreurs et les bugs rencontrés lors des démos. Le lancement de chaque sprint se fait lors d'une réunion \textbf{sprint planning}, dans laquelle toute l'équipe du projet est présente ainsi que le client. Ce dernier désigne les nouvelles \textbf{User Story} et des nouvelles fonctionnalités qu'il souhaite qu'elles soient rajoutées ou corrigées dans la prochaine version de l'application. 

Durant mon alternance, j'ai eu la chance d'être inviter par mes maîtres d'alternances à assister plusieurs réunions des autres projets afin de voir le déroulement d'un projet agile. j'ai pu assister à un sprint planning, une rétrospective et une démo...    

Dans mon cas, j'étais le seul développeur dans mon projet, je faisais un Daily-meeting  tous les jours avec mes maîtres d'alternances pour que je puisse dire sur quoi je travaillais et leurs poser des questions en cas de blocages. L'administrateur des systèmes d'informations(ADSI) JOSSO Clément, qui est en même temps le Product Owner(PO) du projet, et mes maîtres d'alternances me conseillaient sur l'ordre des tâches à suivre lors des multiples réunions.   

\subsubsection{Communication}

Pour communiquer, l'équipe informatique utilise l'outil de messagerie instantané Rocket.Chat. Il permet de créer différents
channels pour chaque équipe, des channels de veille ont été également crées pour de la veille technologique.

Nous utilisons aussi l'outil Calendar (un outil développé par Oracle) qui permet d’organiser les réunions.  

\subsubsection{Dockerisation}

Le logiciel Docker permet de créer, déployer et exécuter des conteneurs de manière efficace. Un conteneur enveloppe l’application d’un logiciel dans une boîte invisible avec tout ce dont il a besoin pour s’exécuter.


Dans le pôle DINAMO, l'équipe utilise Docker Compose pour partager le même environnement de travail entre plusieurs personne et tout installer sur docker. Docker Compose est un outil qui permet de décrire et gérer  plusieurs conteneurs comme un ensemble des services inter-connectés.

Dans les projets sur lesquelles j'ai travaillés, nous avons utilisé Docker-compose pour déclarer tout les conteneurs dont nous avons besoin dans un fichier .yml  et nous démarrons l'ensemble des conteneurs en une seule commande \textbf{docker-compose up}. 

Dans le fichier docker-compose.yml, chaque conteneur est décrit avec un ensemble de paramètres qui correspondent aux options disponibles lors d’un docker run : l’image à utiliser, les volumes à monter, les ports à ouvrir...
 

\subsubsection{Versionning}

L'équipe informatique utilise Git sur la plateforme Gitlab hébergée par la Forge National comme gestionnaire de versions. 

\section{À propos du projet}

\subsection{Contexte et Problématique}

La DSII (direction des systèmes d'information et d'innovation (DSII) et plus précisément l'équipe DNIAMO utilise plusieurs façons pour documenter leur applications. La plupart du temps elle utilise une documentation écrite (Manuel Utilisateur), une chose qui ne rend pas facile aux utilisateurs de se documenter sur l'appli à cause de contrainte de temps. 


C'est pour cette raison que Monsieur JOSSO Clément,le product Owner du projet(PO), a voulu mettre en œuvre une documentation interactive dans l'application métier Solycee afin de faciliter l’accès de la documentation à ces utilisateurs et d'interagir directement avec la page web. Nous avons utilisé un outil nommé BootstrapTour pour réaliser cette documentation. 

Le besoin du client est d'avoir une application qui permet à l'ADSI(Administrateur Des Systèmes d'Informations) de créer la documentation directement à partir de sa page web et pouvoir l'améliorer au fur et à mesure des demandes des utilisateurs via une interface graphique rajouter dans le projet Solycee. 


Le but final du projet est de permettre aux ADSI d'être autonomes et de pouvoir rajouter la documentation dans leurs applications. Nous avons donc crée un web service(AppTour) où sont sauvegardés tous les tours et leurs étapes.
 
\subsection{Présentation de Bootstrap Tour : Une visite guidée interactive}
 
Bootstrap tour est une librairie JavaScript permettant de présenter de façon assez simple l’application, l’avantage, comme pour toutes les librairies en fait, c’est qu’il y a déjà des bouts de fonctionnalités toutes faites, ce qui évite de réimplémenter le code. Bootstrap tour est une implémentation qui sert à créer une visite guidée pour une application.

Le bénéfice de l'utilisation de Bootstrap tour est immédiat pour les utilisateurs car elle présente les fonctionnalités de la page directement sur l'écran soit en lançant la documentation  dés l'ouverture de la page ou en appuyant sur un bouton pour déclencher cette documentation.


\subsection{Architecture du projet}

Le projet se constitue d'une interface utilisateur qui est une application web (Solycee) et d'un Back-end qui est une API REST (AppTour) qui communique avec une base de données relationnelles. Voir figure 2.


\begin{figure}[H]
	\centering
 		\includegraphics[width=1\textwidth]{diagrammes/ArchitectureGenerale.png}
  		\caption{Architecture du projet}
	\end{figure}


\begin{enumerate}
\item \textbf{Technologies :}\\

\begin{itemize}
\item \textbf{Java: }Langage du développement du Back-end de Solycee. 
\item \textbf{Kotlin: }Langage du développement de l'API Rest AppTour.
\item \textbf{Maven: }Construction des projets Java et Kotlin du Back-end.
\item \textbf{Spring: }Spring est un framework libre pour construire et définir l'infrastructure d'une application java dont il facilite le développement et les tests
\item \textbf{Spring Boot: }Framework d'infrastructure des Back-end.
\item \textbf{Hibernate: } Framweork 
\item \textbf{Spring Security: }Spring Security est un Framework de sécurité léger qui fournit une authentification et un support d’autorisation afin de sécuriser les applications Spring.
\item \textbf{JavaScript,CSS,HTML: } Développement du Front-end de Solycee.
\end{itemize} 

\end{enumerate}

\subsubsection{Back-end}

Le Back-End, c’est la partie du code qui est exécutée par le serveur, il s’agît d'un serveur fournissant une API gérant la persistance des données et la logique de l'application.La majorité des applications du rectorat sont développées en framework Spring-boot et en langage Java et il existe quelques applications développées en langage Kotlin dont l'application Apptour.

L'architecture des projets du rectorat est unique dans tous les projets afin  de faciliter la prise en main de ces derniers et avoir le même point de vue au sein du l'équipe.

Le Back-end des applications se décompose de plusieurs couches: \newline

Une couche Repository/DAO: Repositories sont des interfaces héritant de l'interface Repository. L'objectif de ces interfaces consiste à rendre la création de la couche d'accès aux données (requêtes SELECT, UPDATE...) plus rapide.\newline


Une couche Service: Elle permet de séparer les opérations effectuées par le contrôleur et celles qui concernent le modèle des données. Toute action devrait passer par cette couche.\newline


Une couche Contrôller: Il s'agit de l'API qui permet de répondre à toutes les requêtes envoyées par le client. 

\begin{figure}[H]
	\centering
 		\includegraphics[width=1\textwidth]{diagrammes/ArchitectureProjet.png}
  		\caption{Arichtecture Back-end d'une application web}
	\end{figure}

\section{Objectifs et missions de l'alternance}

\subsection{Solycee}

Solycee est une application métier créée par le pôle DINAMO et destinée à la gestion des offres et demandes de stages. Elle permet aux collèges et aux lycées d'inscrire leurs élèves dans des mini-stages proposés par les lycées.  

L'application serveur est réalisé en Java avec l'utilisation des composants du framework Spring.L'interface est codé avec le triplet HTML,CSS \& Javascript en intégrant du Thymeleaf pour le traitement des vues.  

\subsubsection{Aspect technique}

Le projet est implémenté selon le patron de conception MVC et réalisé avec plusieurs technologies, la plupart de ces dernières sont open-source.L'application est implémenté avec un back-end et un Front-end.

Le back-end a été implémenté avec le framework Spring et le langage Java. Le coté interface (Front-end) a été réalisé avec le moteurs de templâte Thymeleaf. 

Voici maintenant une présentation plus précise sur les principaux outils utilisés:\newline

\textbf{Spring:}  Spring est un framework libre pour construire et définir l'infrastructure d'une application java dont il facilite le développement et les tests.\newline

\begin{itemize}
	\item \textbf{Spring MVC} : Spring MVC est un framwork qui permet d’implémenter des applications selon le design pattern MVC. Donc, comme tous autre MVC framework, Spring MVC se base sur le principe décrit par le schéma ci-dessous :\newline
	

\begin{figure}[H]
	\centering
 		\includegraphics[width=1\textwidth]{diagrammes/mvc.jpg} 
  		\caption{Schéma du patron de conception MVC}
	\end{figure}
	
	\item \textbf{Spring Security }: Spring Security est un Framework de sécurité léger qui fournit une authentification et un support d’autorisation afin de sécuriser les applications Spring.\newline
\end{itemize} 

\textbf{Maven:} Maven est un outil de construction de projets (build) open source développé par la fondation Apache. Il permet de faciliter et d'automatiser certaines tâches de la gestion d'un projet Java.\newline

\textbf{Thymeleaf:} Thymeleaf est un  Java XML/XHTML/HTML5 Template Engine qui peut travailler à la fois dans des environnements Web (Servlet) et celui de non Web. Il est mieux adapté pour diffuser XHTML/HTML5 sur View (View Layer) des applications Web basées sur MVC. Mais il peut traiter n'importe quel fichier XML même dans les environnements hors ligne. Il fournit une intégration complète de Spring Framework. 

\subsubsection{Travail réalisé}

Dans le projet Solycee, j'ai principalement travaillé sur la partie Front-end. J'ai réalisé plusieurs tâches soit en rajoutant des nouvelles vue et des nouveaux composants dans les pages web et/ou en rajoutant des nouvelles fonctionnalités sur les composants existant.

Le projet a été développé par l'équipe Dinamo. Avant de commencer à réaliser les tâches demandées, il était nécessaire de procéder un travail  de lecture du code implémenté et de la compréhension de ce dernier en posant des multiples questions à mes maîtres d'alternances. Il m'a fallut aussi comprendre l'architecture du projet ainsi que les technologies utilisées. 

Une fois que le code était compréhensible et j'ai acquis toutes les informations importantes sur le projet, j'ai commencé à réalisé les tâches demandées.  Voici une présentation plus précises de toutes les tâches réalisées:   


\begin{itemize}
\item Rajout d'une modale pour prévenir les utilisateurs de la présence d'un bouton aide pour déclencher la documentation interactif.

\begin{figure}[H]
	\centering
 		\includegraphics[width=1\textwidth]{diagrammes/aide_modal.png}
  		\caption{Image page d'accueil Solycee}
	\end{figure}
	
Pour que ça soit pas dérangeant pour les utilisateurs qu'ils ont déjà vu la modale et qui ont eu l'info, j'ai fait en sorte que ça s'affiche qu'une seule fois pour eux. 
   
\item Rajout du Bootstrap Tour : Il s'agit de l'ajout d'un bouton \includegraphics[width=5mm,scale=0.5]{diagrammes/Bouton_aideDispo.png} sur toutes les pages web de l'application. En appuyant dessus ça déclenche les pop-ups  de Bootstrap tour (Les étapes du tour sur la page web), une étape se représente avec 3 champs: Element (l'id d'élément cliqué, Titre (titre de l'étape) \& Content (Description de l'étape). Dans le cas où il n'existe pas de tour sur la page web, le bouton affiché est gris et titré par une phrase "Aide non disponible pour cette page" \includegraphics[width=7mm,scale=0.5]{diagrammes/Bouton_aideNonDispo.png}  

\begin{figure}[H]
	\centering
 		\includegraphics[width=1\textwidth]{diagrammes/exemple_Tour.png} 
  		\caption{Exemple d'une étape de Bootstrap Tour}
	\end{figure}

\item Récupérer le tour d'une page web avec ses étapes: Pour chaque page web de solycee, l'application récupère le tour de cette dernière avec ses étapes en lançant une requête à l'API AppTour en passant en paramètre le nom de l'application et l'identifiant de la page courante. Cette requête (FindTourWithLibelleAplicationAndIdHtmlPage) renvois un json avec toutes les étapes en cas de la présence du tour dans la base. En cas où le tour n'existe pas et la requête renvoi une réponse 404 (Not found), nous mettons le bouton d'aide en gris et il devient non cliquable. 

\begin{figure}[H]
	\centering
 		\includegraphics[width=1\textwidth]{diagrammes/aideNonDispo.png} 
  		\caption{Pas de tour disponible dans ce page Web }
	\end{figure}
    

\item Mode édition pour les Administrateurs : En appuyant sur le bouton  \includegraphics[width=5mm,scale=0.5]{diagrammes/Bouton_modeEdition.png}, les administrateurs  de l'application passent en mode édition. Il s'agit d'un mode avec la même interface graphique mais les composants et les boutons de cette interface ne réagissent plus de la même façon(Le fonctionnement de ces composant se désactive). En appuyant sur un des composant de la page, il y a une modale qui s'affiche pour permettre aux admins de rajouter des étapes de leur tour dans cette page Web.  Nous affichons aussi un champs sur l'interface Web pour afin de prévenir les admins qu'ils ont passé en mode édition.

 \begin{figure}[H]
	\centering
 		\includegraphics[width=1\textwidth]{diagrammes/mode_edition.png} 
  		\caption{Page d'accueil en mode édition}
	\end{figure}

\item Permettre seuls aux administrateurs de voir le bouton qui permet de passer en mode édition. En se connectant à l'application, nous récupérons le nom complet de l'utilisateur connecté et nous envoyons une requête à l'API AppTour pour voir si l'utilisateur est administrateur. Si la requête renvoie le json de l’utilisateur nous affichons le bouton, dans le cas inverse le bouton ne sera pas affiché. \newline 

\item Rajouter ou modifier les étapes :Quand les administrateurs se mettent en mode édition et en appuyant sur les composant de la page web(Boutons, liens ou autres), une modale s'affiche afin de permettre aux administrateurs de rajouter une étape du tour dans le cas où aucune étape existe pour le composant appuyé ou modifier l'étape si elle existe. 
La récupération de l'étape de chaque composant se fait par un appel à l'API AppTour (FindStepByElementandIdTour), dans le cas où l'étape existe, nous remplissons les champs de la modale avec les données récupérées par la requêtes(Title \& content). Dans le cas inverse, nous laissons les champs vide afin que les administrateurs remplissent les données qui veulent stocker dans la base AppTour.

Il existe un bouton valider dans la modal qui  permet soit de rajouter la nouvelle méthode avec la méthode POST si elle n'existe pas ou sinon modifier les éléments de l'étape avec une méthode PUT. 
\end{itemize}

\subsection{AppTour}   

Au début du projet, j'ai commencé par manipuler BootstrapTour sur l'application Solycee en rajoutant le code de la librairie dans le front de Solycee (code en javascript avec les étapes souhaités). Bootstrap tour permet d'afficher l'étape du tour et aussi passer à l'étape suivante ou précédente ainsi que arrêter le tour avec le bouton "Fin".

Cette manipulation est simple et efficace pour un développeur pour rajouter des tours et leurs étapes afin de faire une visite d'application. Il suffit juste de manipuler le code Javascript. Chose qui n'est pas facile pour d'autres personnes ou les administrateurs de l'application.

AppTour est un web service qui permet aux administrateurs des applications du rectorat d'être plus autonome afin qu'il puissent rajouter ou/et modifier des étapes sans aller chercher dans le code. Tous les tours et les étapes de toutes les applications où nous souhaitons mettre du Bootstrap tour seront sauvegardés dans le web service AppTour.   

API Apptour est une application Sring Boot codé en langage Kotlin. Le but de cette application est d’implémenter plusieurs  services afin de permettre aux administrateurs des autres applications de rajouter et modifier les étapes. 


\subsubsection{Aspect technique}
 
 Le projet AppTour était initialiser par l'alternant de l'année dernière. j'ai commencé par la relecture du code et la compréhension de l'existant. L'application a été implémenté en Kotlin qui est un langage de programmation orienté objet et fonctionnel, avec un typage statique qui permet de compiler pour la machine virtuelle Java et JavaScript et Spring boot qui est un framework utilisé afin de faciliter la configuration d'un projet Spring et de réduire le temps alloué au démarrage d'un projet. plusieurs autres technologies était utilisées dans ce projet : 
 
\begin{itemize}
\item Maven :  Est un outil de gestion et d'automatisation de production des projets logiciels Java en général et Java EE en particulier. Il est utilisé pour automatiser l'intégration continue lors d'un développement de logiciel.

\item JUnit : Pendant mon développement de l'API AppTour, j'ai utilisé la méthode du TDD(Test Driven Development),le développement piloté par les tests. j'ai utilisé le framework JUnit pour implémenter les tests unitaires afin d'assurer la maintenance et l'efficacité de l’application.

\item Git : Est un logiciel de gestion de versions décentralisé. Il permet de sauvegarder les fichiers en gardant la chronologie de chaque sauvegarde.

\item L'API AppTour a été réalisé de la même manière que les autres applications du rectorat. Il existe plusieurs couches. Une couche service pour implémenter tous les services dont nous avons besoin, une couche controller pour construire l'API utilisée par le Front Solycee et qui sera utilisée par d'autres applications dans le futur, il existe d'autres couches afin de créer une application similaire aux autres applis du rectorat.

\end{itemize}

\subsubsection{Travail réalisé}

Dans le projet AppTour, j'ai travaillé sur la réalisation d'une API web (Le back-end pour le bootstrapt), j'ai développé plusieurs Web Services qui permets de rajouter ou modifier les étapes des tours ainsi que faire des recherches pour vérifier l’existence des tours. 

Comme noté plus haut, le projet AppTour était initialiser par l'alternant de l'année dernière en créant les tables (Entity). Il existait deux tables, une table Tour et une table Step. j'ai commencé donc par la relecture du code existant et la modification des tables.
Ensuite, Nous avons travaillés avec mes maîtres d'alternance sur l'architecture du projet.

Une fois que le code et le travail demandé était compris et acquis, j'ai commencé par implémenté du code nécessaire pour réaliser les web services dont nous en avons besoin. Voici une présentation plus précises de toutes les tâches réalisées:   

\begin{itemize}

\item rajouter des éléments dans les tables par exemple Storage dans l'entité Tour qui est un booléen et qui permet de configurer si on veut stocker les tours dans l'historique ou pas ou encore la variable Ordre dans l'entité Step qui un entier et qui permet de donner un ordre pour chaque step.  
\item Créer toutes les classes DTO afin de les utiliser dans les requêtes  pour envoyer ou récupérer des données.

\item Implémenter les classes Repositories qui permettent de faciliter l'accès  ou le stockage des données dans la base.

\item Créer des classe Mapper qui permettent de convertir un objet Dto à un objet normal ou l'inverser. Par exemple passer d'un objet TourDto à un objet Tour il suffit d'appeler la méthode tourDtoToTour et qui prend en paramètre un objet tourDto : TourDto.  

 \begin{figure}[H]
	\centering
 		\includegraphics[width=1\textwidth]{diagrammes/Exemple_tourDtoToTour.png} 
  		\caption{Exemple d'une méthode de la classe TourMapper}
	\end{figure}

\item La réalisation de l'API et tout les services Web, il s'agit d’implémenter des classes contrôllers qui consomme toute les méthodes qui se trouvent dans les classes services. Il existe plusieurs types de services web: 
\begin{itemize}
\item Des méthodes GET : Pour la récupération des données de la base dans le cas où elles existent. Par exemple "retrieveTourByIdHtmlPage" ou encore "getStepByElementandIdTour".

\item Des méthodes POST : Pour stocker des nouveaux objets dans la bas de données. Par exemple "saveTour" ou encore "saveNewStep".

\item Des méthodes PUT : Pour mon projet, j'ai du m’en servir  d'une méthode "updateStep" qui permet au utilisateurs de modifier une étapes qui existe déjà.

\end{itemize}
\item L'implémentation des classes Services: Toutes les actions effectuer le contrôller doivent passer par les classes services. 

Il existe une classe TourService: Dans cette classe, j'ai réalisé plusieurs méthodes qui permettent à stocker ou récupérer des Tours de la base données : 
\begin{itemize}
\item  retrieveTourByLibelleAndIdHtmlPage : Elle permet à récupérer un tour en passant en paramètre nom d'une application et l'identifiant de la page HTMl. 

\item saveTourWithLibelleAndIdHtmlPage : Elle permet à stocker un tour dans la base de données en donnant le nom de l'application et l'Id de la page HTML. 

\item updateTour : Elle permet à modifier un tour existant dans la base données en donnant des nouveaux paramètres. 
\end{itemize}

Il existe aussi une classe StepService : Dans cette classe, j'ai réalisé plusieurs méthodes qui permettent à stocker ou récupérer des Step de la base données : 
\begin{itemize}

\item  findStepByElementandIdTour : Elle permet de récupérer un Step en passant en paramètre l'élément rechercher et l'identifiant du tour.

\item saveStep : Elle permet à stocker un Step dans la base de données.

\item findOrdreMax : Elle permet de lister tout les step d'un tour par ordre croissant.  
\end{itemize}
\end{itemize}
\newpage
\subsection{Partie Sécurité}
\subsubsection{Permettre seuls aux administrateurs de modifier les tours}

Pour permettre seuls aux administrateurs de l'application Solycee de passer en mode édition  et pouvoir rajouter et/ou modifier des tours, j'ai créé une nouvelle table(Entité) Personnel dans l'API AppTour, cette table contient un identifiant et libellé de l'application, les personnes présentes dans la tables Personnel sont des administrateurs de leurs applications. Après j'ai commencé à réaliser les classes nécessaires dans la même architecture que les tables Tours et Step. J'ai implémenté donc une classe PersonnelDto et puis une classe PersonnelMapper qui permet de passer d'un objet Personnel à un objet PersonnelDto ou l'inverse(d'un objet PersonnelDto à un objet Personnel), ensuite j'ai créé la classe PersonnelService pour pouvoir implémenter les méthodes utilisées par la classe PersonnelController.Enfin, J'ai implémenté un service qui permet de faire une recherche dans la table Personnel et vérifier la présence de la personne en donnant son identifiant (Nom + Prénom) et aussi le libellé de l'application. 


En laçant l'application Solycee, je récupère l'identifiant de la personne connecté (Toute personne connecté a un identifiant (Nom + Prénom)) et puis j’envoie une requête à l'API AppTour pour vérifier si la personne connecté est présente dans la table Personnel. Dans le cas ou la requête renvoie une réponse positive (200 Succès), j'affiche le bouton qui permet aux admins de passer en mode édition, dans le cas contraire le bouton \includegraphics[width=5mm,scale=0.5]{diagrammes/Bouton_modeEdition.png} ne s'affiche pas. 



\subsubsection{Sécuriser l'API}


Après avoir réaliser toutes les tâches demandés par le PO(Product Owner), il voulait mettre les nouvelles fonctionnalités rajoutées en production pour avoir un avis des utilisateurs. Mais avant de les mettre en prod fallait sécuriser l'API Apptour ainsi que s'assurer que les personnes qui voudraient envoyer des requêtes à l'API ont les droit de réaliser ces requêtes et ils sont administrateurs de l'application. Nous avons donc décidé de sécuriser les échanges entre le client (Solycee) et l'API AppTour. 

Le premier travail a commencé par faire des recherches par rapport à la façon dont nous allons sécuriser les applications.Un sujet de recherche du rectorat était de pourvoir intégrer la plateforme Keycloak pour l'authentification des utilisateurs dans les applications. Solycee était une occasion pour commencer les recherches sur ce sujet.  Nous avons décidé de mettre en place une couche d'authentification basé sur OAuth 2.0 qui permet d'autoriser une application web (ou un logiciel) à utiliser une autre API sécurisé, et nous avons utilisé Keycloak localement pour s'authentifier et obtenir un Token JWT pour chaque utilisateurs de l'application.  \includegraphics[width=30mm,scale=0.5]{diagrammes/logo_Keycloak.jpeg}  

La première étape était  d'installer un serveur Keycloak localement qui permettra par la suite de faire des tests. Keycloak  est un logiciel open source permettant l'authentification unique avec la gestion des identités et des accès, destiné aux applications et services modernes. Pour la configuration de keycloak, il suffit juste de fournir un fichier  royaume(Realm) ainsi qu'un fichier utilisateurs(User) pour renseigner les utilisateurs  avec leur identifiants et leurs mots de passe (Voir annexes pour un exemple des fichiers Realm et User)

Le fonctionnement de Keycloack nécessite  une configuration. IL faut déclarer plusieurs données : 
\begin{itemize}
\item Utilisateurs : Ils sont des entités pouvant se connecter à votre système
\item Rôles : ils identifient un type ou une catégorie d’utilisateur ( Admin, user, …)
\item Groupe : permet de gérer un groupe d’utilisateur
\item Royaumes (realms): un domaine qui gère un ensemble d’utilisateurs, d’informations d’identification, des rôles et des groupes.
\item Clients : ils sont des entités pouvant demander à Keycloak d’authentifier un utilisateur ( comme des applications )
\end{itemize}

Après l’authentification, nous obtenons un Identity token qui fournit des informations d’identité sur l’utilisateurAccess token : un token pouvant être fourni dans le cadre d’une requête HTTP autorisant l’accès au service invoqué.

En suite, j'ai fait des recherches pour trouver une libraire JavaScript qui permet à utiliser keycloak pour l'authentification. Pendant mon développement j'ai utilisé deux librairie JavaScript différentes mais le fonctionnement était le même. J'ai commencé par implémenté la librairie KeycloakJS et par la suite j'ai utilisé la librairie Oidc-clientJS pour permettre aux administrateurs de se diriger à une plateforme d'authentification, dans mon cas c'était Keycloak pour tester la librairies. Voici la plateforme d'authentification Keycloak: 
 \begin{figure}[H]
	\centering
 		\includegraphics[width=1\textwidth]{diagrammes/authentification_keycloak.png} 
  		\caption{Authentification avec OpenID et Keycloak }
	\end{figure}

\newpage
\section{Annexes}

Nous trouverons dans cette partie tout les éléments graphiques, schémas, diagrammes,parties de code  qui n’ont pas pu être ajouté au rapport faute de place.

\end{document}
